
%----------------------------------------------------------------------------------------
%	PACKAGES AND OTHER DOCUMENT CONFIGURATIONS
%----------------------------------------------------------------------------------------

\documentclass[a4paper, 11pt]{article} % Font size (can be 10pt, 11pt or 12pt) and paper size (remove a4paper for US letter paper)
\usepackage[protrusion=true,expansion=true]{microtype} % Better typography
\usepackage{graphicx} % Required for including pictures
\usepackage{wrapfig} % Allows in-line images
\usepackage[
    type={CC},
    modifier={by-nc},
    version={4.0},
]{doclicense}
\usepackage{mathpazo} % Use the Palatino font
\usepackage[T1]{fontenc} % Required for accented characters
\linespread{1.05} % Change line spacing here, Palatino benefits from a slight increase by default

\makeatletter
\renewcommand\@biblabel[1]{\textbf{#1.}} % Change the square brackets for each bibliography item from '[1]' to '1.'
\renewcommand{\@listI}{\itemsep=0pt} % Reduce the space between items in the itemize and enumerate environments and the bibliography

\renewcommand{\maketitle}{ % Customize the title - do not edit title and author name here, see the TITLE block below
\begin{flushright} % Right align
{\LARGE\@title} % Increase the font size of the title

\vspace{50pt} % Some vertical space between the title and author name

{\large\@author} % Author name
\\\@date % Date

\vspace{40pt} % Some vertical space between the author block and abstract
\end{flushright}
}

%----------------------------------------------------------------------------------------
%	TITLE
%----------------------------------------------------------------------------------------

\title{\textbf{Does reductionism succeed in explaining how we can know things on the basis of testimony?}\\ % Title
} 

\author{\textsc{Stephen Whitenstall} % Author
\\{\textit{The Open University}}} % Institution

\date{April 2015}

%----------------------------------------------------------------------------------------

\begin{document}

\maketitle % Print the title section

%----------------------------------------------------------------------------------------
%	ABSTRACT AND KEYWORDS
%----------------------------------------------------------------------------------------

%\renewcommand{\abstractname}{Summary} % Uncomment to change the name of the abstract to something else

\begin{abstract}
In this essay I will address the question of whether reductionism is successful as an explanatory account of how people can know things on the basis of testimony. I will consider versions of anti-reductionism and reductionism from Thomas Reid, David Hume and Elizabeth Fricker, and assess whether Paul Faulkner's and Alan Millar's approaches to testimony are forms of reductionism. Finally I will conclude that different forms of reductionism can only provide partial explanations for different types of testimony.
\end{abstract}

\hspace*{3,6mm}\textit{Keywords:} reductionism , testimony % Keywords

\vspace{30pt} % Some vertical space between the abstract and first section

%----------------------------------------------------------------------------------------
%	ESSAY BODY
%----------------------------------------------------------------------------------------

\section*{}

The philosophical debate around testimony is broadly concerned with whether testimony can be knowledge in itself or whether it can only be a conveyance of knowledge. A distinction can be drawn in this context between the primacy of knowledge as a true and direct representation of things in themselves and subsidiary belief as being true by virtue of being justified by reasons or having a causal connection to the thing in itself (Goldman, 2005) \cite{Goldman:2005}. The most general definition of reductionism in respect of testimony is the view that testimony cannot be knowledge in itself, but rather must be ''defined or explained'' by some subsidiary source of information (p.194, Price/Chimmiso, 2014) \cite{Price/Chismisso:2014} and it is this version of the reductionist view that I will be focussing on in the discussion that follows.

%------------------------------------------------

\section*{Thomas Reid's anti-reductionism}

If testimony were knowledge in itself, there would be no need to refer to anything else in order to accept it as knowledge. Such an approach, referred to as \textit{anti-reductionism}, treats testimony as sufficient grounds in itself to justify belief, based entirely on the content of what is attested. So in an anti-reductive scenario, one can know 'that p' exclusively and directly from another person who has attested 'that p'. Thomas Reid argued that humans have an innate propensity for veracious testimony (the principle of veracity), together with an innate disposition to believe the testimony of others (the principle of credulity) (p.176ab, Price/Chimmiso, 2014)\cite{Price/Chismisso:2014}. The principle of credulity can be taken to be a form of the principle of charity (Mackie, 2005)\cite{Mackie:2005}, where scepticism is minimised in deference to beliefs that are ''guided by the authority and reason of others'' (p.177a, Price/Chimmiso, 2014)\cite{Price/Chismisso:2014}. A problem for Reid's anti-reductionism is that he failed to explain how a person might escape naturalised credulity, or be ''brought to maturity by proper culture'', or how they might ''set bounds to [\dots] authority'' when they are, in effect, already bound themselves by testimonial authority (p.177b, ibid)\cite{Price/Chismisso:2014}.

%------------------------------------------------

\section*{David Hume's reductionism}

In contrast to Reid's anti-reductionism, David Hume's reductionism argued that, for testimony to be knowledge there must be some positive reason to believe what is attested that reduces or refers to something other than the content of the testimony itself. So in a reductive scenario, one cannot know 'that p' exclusively and directly from testimony; instead one must use some form of \textit{inference} that serves as a further justification 'that p'. Reid's anti-reductionism allowed that testimony be interpreted \textit{a priori} as direct evidence. In contrast Hume's reductionism required that testimony could only be interpreted as knowledge \textit{a posteriori} via customs of indirect inference (E 10.8, Hume, 1748, 1777)\cite{Hume:1748}.

\vspace{10pt}

For Hume, the ''ultimate standard'' by which beliefs (testimonial or otherwise) could be justifiably inferred always reduced to ''experience and observation'' (p.174b, Hume, 2007/1736)\cite{Hume:2007}. Unlike Reid, Hume did not allow for any principle of credulity to favour testimony. On the contrary, he emphasised the contingency of testimony, where what is attested is only regarded ''as a proof or a probability'' when in conjunction with ''any kind of object'' that stands apart from the testimony itself (p.174a, ibid)\cite{Hume:2007}. An object, in this context, is an epistemic object of enquiry; this may take the abstract, deductive form of a proof, or the probabilistic form which generalises from inferring causal connections from what is experienced or observed. For Hume, some kind of conjunction between epistemic objects was necessary to justify knowledge, because he believed that no object in itself ''...implies the existence of any other...'' (p.61a, Hume, 2007)\cite{Hume:2007}. 

\section*{Coady's interdependence objection to Hume's reductionism}

The \textit{interdependence objection} to Hume's reductionism argues that there cannot be any independent object of reference from which the reliability of testimony can be checked (p.103b, Price/Chimmiso, 2014)\cite{Price/Chismisso:2014}, because the way we experience and observe the world must be, in part at least, also based upon testimony (p.103, ibid). C. A. J. Coady highlighted what he saw as ''a fatal ambiguity'' in Hume's use of the terms experience and observation (p.80b, Coady, 1994)\cite{Coady:1994}. But Hume was quite explicit and unambiguous when he recognised that ''... this experience is not entirely uniform on any side ...'' (p.174c, Price/Chimmiso, 2014)\cite{Price/Chismisso:2014} and that assurance in testimony is a matter of balancing the ''opposite circumstances'' of ''our judgements'' and ''the reports of others'' (ibid)\cite{Price/Chismisso:2014}. 

\vspace{10pt}

Another objection to Hume's reductionism is that, in requiring one to always possess further positive reasons to ground justification for belief in testimony, it sets too high a standard (p.103a, ibid)\cite{Price/Chismisso:2014}. The difficulty of meeting such a standard might lead to a general scepticism about testimony being a source of belief or knowledge at all (p.394, Fricker, 1995)\cite{Fricker:1995}. But Hume was sceptical that knowledge has a definitive foundation, and only allowed in his Treatise (1736) that beliefs be contingently justified from experience (p.61b, Hume, 2007)\cite{Hume:2007}. 

\section*{Fricker's arguement against Coady's interdependence objection}

Elizabeth Fricker argued against Coady's assumption that reductionism necessarily implies the need for a global confirmation of testimony (p.403c, Fricker, 1995)\cite{Fricker:1995}. Contrary to Coady's characterisation of reductionism, Fricker highlighted the possibility of local reductionism,  where testimony may be evaluated as trustworthy given ''adequate grounds'' for particular assessments of discrete utterances (p.404, ibid)\cite{Fricker:1995}. However, Fricker qualified the possibility of local reductionism by limiting such assessments to a ''mature phase'' of human development (p.403a, ibid)\cite{Fricker:1995}, and she conceded that testimony is ''simply-trusted'' and treated non-reductively as direct knowledge at a developmental stage (p.403b, ibid)\cite{Fricker:1995}.

\section*{Fricker's local reductionism}

Such a qualification might seem to suggest that we can only know things on the basis of testimony non-reductively, therefore directly, at a developmental stage; and reductively, therefore by inference, at a mature stage. Fricker's local reductionism might then also be criticised with the aforementioned interdependence objection: that our mature judgement of testimony is ultimately based upon a developmental credulity. Fricker counters this objection by arguing that testimony taken on trust while we develop only endures if it remains consistent with mature testimony and with what we perceive as we mature (p.410, ibid)\cite{Fricker:1995}. The interdependence objection depends on an extreme form of holism (Heil, 2005)\cite{Heil:2005}, where testimony is treated as a unitary category that is determined by relations between testimony, experience and perception. But once testimony is disaggregated into different categories (p.407, Fricker, 1995)\cite{Fricker:1995}, then the interdependence between testimony, experience and perception is moderated both by an incompleteness of development/maturity and varying degrees of justification. However, such a disaggregation also undermines the success of a straightforward or unitary explanation of testimony in reductionist terms.

\section*{Paul Faulkner's assurance approach to testimony}

A reductionist explanation need not be limited only to the theoretical possibility of relating what is attested to something further that makes it true; it might also address the practical matter of how ''a speaker telling an audience something'' may result in ''the audience believing the speaker'' (p.877, Faulkner, 2007)\cite{Faulkner:2007}. Any reduction here is not only to evidence of truth, but also to an assurance that testimony indicates truth (p.878, ibid)\cite{Faulkner:2007}. Paul Faulkner's \textit{assurance approach} to testimony highlights presumptive relationships of trust that (if efficacious) serve to reinforce not evidential truth but some social responsibility in situations where an audience is dependent upon a speaker to tell the truth. Faulkner makes a distinction between predictive trust that is ''justified on the basis  of the evidence'' (p.885, ibid)\cite{Faulkner:2007} and affective trust that is justified by a grounding presumption of belief (p.887a, ibid)\cite{Faulkner:2007}. From a reductionist perspective, predictive trust reduces to external, prior evidence, whilst affective trust depends upon ''a subject's [internal] reasons for acting or holding [an] attitude'' (p.887b, ibid)\cite{Faulkner:2007}. In my view, it is ambiguous whether affective trust reduces at all to a further external source, as it is conditional upon the internal, presumptive beliefs of a subject who chooses to trust (p.888, ibid)\cite{Faulkner:2007}. 

\section*{}

Faulkner suggests that affective trust is based upon a disposition to accept, rather than believe, testimony (p.893, ibid)\cite{Faulkner:2007}. Such a disposition is enacted within a virtuous circle of trusted relationships, shared values and desires (p.895, ibid)\cite{Faulkner:2007}. Consequently testimony is effectively trusted not by reduction to secondary reasons but as the performance of ''a basic and non-reducible attitude'' (p.896, ibid)\cite{Faulkner:2007}. Faulkner's analysis emphasises the social nature of testimony as \textit{knowing-how} to maintain normative conventions of evaluation (p.890, ibid). As Gibert Ryle argued, ''knowing a rule is not a case of knowing an extra fact or truth'' (p.227, Ryle, 2009)\cite{Ryle:2009}. Thus any explanation of how we can know things via testimony should include \textit{knowing-how}, not just \textit{knowing-that}. Trusting a stranger to direct you to the station is primarily a matter of \textit{knowing-how} to ''accept her assurance'' (p.109, Price/Chimmiso, 2014)\cite{Price/Chismisso:2014}, and cannot just be a matter of \textit{knowing-that} her assurance reduces \textit{a posteriori} to a static, factual confirmation. 

\section*{}

If testimony is a form of \textit{knowing-how} (to trust or accept rules), then it may be explained as either (1) a non-reductive, \textit{prima facie} indication of meaning or (2) a reduction to some basic mechanism that may explain how rules of trust work to produce knowledge. However, stubbornly pursuing a justification for testimony that simply reduces to accepting a principle of trust is analogous to Achilles telling the Tortoise that ''...Logic would force you to do it'' [to accept the principle of inference] (p.693, Carroll, 1995/1895)\cite{Carroll:1995}, and would lead to an infinite regress of premises or reasons. In effect, when a stranger's directions are simply accepted, it is as if they are asserted and inferred as true based upon \textit{knowing-how} a principle of trust works. Furthermore only the antecedent premises are available, \textit{knowing-that} the consequent truth of the directions is not possible empirically, and cannot be further deduced if the directions (and the rules of trust) are to be followed.

\section*{}

Alan Millar suggests that knowledge can be acquired from the \textit{knowing-how} of ''recognitional abilities'', and does not necessarily ''result from reasoning'' (p.184, Millar, 2010)\cite{Millar:2010} or \textit{knowing-that}. So knowing from being told something reduces to an ability to recognise and respond to features extraneous to the content of an utterance, that are ''truth-indicative'' (p.188a, ibid)\cite{Millar:2010}. Millar highlights straightforward cases of telling where knowledge can be obtained solely on the basis of the teller being trustworthy, and the recipient knowing the teller sufficiently to recognise their sincerity and competence given the context (p.188b, ibid)\cite{Millar:2010}. Millar gives an example of such a straightforward case where Millar knows his son is competent in knowing-that a fact (that a particular book is in Millar's study), and his son is sincere in his belief that he knows the fact and intends to communicate his knowledge to his father truthfully (p.189, ibid).  In my view, however, it is ambiguous whether Millar's straightforward cases of testimony are knowledge at all if they only amount to ''... little more than an ability to recall or acknowledge a publicly available known fact'' (p.193, ibid)\cite{Millar:2010}; and whether such knowledge is limited to a social domain of ''repeated encounters with sources of information'' (ibid)\cite{Millar:2010}, rather than seeking further evidence for its own sake through enquiry.

\section*{Conclusion}

In conclusion, some form of reductionism is necessary to explain how things can be known on the basis of testimony. Even if some types of testimony are similar to perception, and directly convey certain forms of knowledge, any explanation of this process would necessarily be an external account of \textit{knowing-how} from a further source. Such an account of \textit{knowing-how} to receive or attest would primarily be descriptive, and would not necessarily justify direct knowledge with reasons, but rather explain a causal connection. Where cases of testimony are not straightforward then a recipient may either infer what the case is themselves, or else delegate such work to ''expert speakers'' (p.705, Putnam, 1973)\cite{Putnam:1973}. But overall there is no homogeneous form of reductionism that can account for all types of testimony. 

%----------------------------------------------------------------------------------------
%	BIBLIOGRAPHY
%----------------------------------------------------------------------------------------

\bibliographystyle{unsrt}
\bibliography{2015-04-OU-TMA05-A333}

%----------------------------------------------------------------------------------------

\doclicenseThis

\end{document}