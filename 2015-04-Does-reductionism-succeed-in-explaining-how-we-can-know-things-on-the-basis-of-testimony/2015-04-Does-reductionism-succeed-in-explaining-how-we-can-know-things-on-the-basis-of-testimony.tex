
%----------------------------------------------------------------------------------------
%	PACKAGES AND OTHER DOCUMENT CONFIGURATIONS
%----------------------------------------------------------------------------------------

\documentclass[a4paper, 11pt]{article} % Font size (can be 10pt, 11pt or 12pt) and paper size (remove a4paper for US letter paper)

\usepackage[protrusion=true,expansion=true]{microtype} % Better typography
\usepackage{graphicx} % Required for including pictures
\usepackage{wrapfig} % Allows in-line images

\usepackage{mathpazo} % Use the Palatino font
\usepackage[T1]{fontenc} % Required for accented characters
\linespread{1.05} % Change line spacing here, Palatino benefits from a slight increase by default

\makeatletter
\renewcommand\@biblabel[1]{\textbf{#1.}} % Change the square brackets for each bibliography item from '[1]' to '1.'
\renewcommand{\@listI}{\itemsep=0pt} % Reduce the space between items in the itemize and enumerate environments and the bibliography

\renewcommand{\maketitle}{ % Customize the title - do not edit title and author name here, see the TITLE block below
\begin{flushright} % Right align
{\LARGE\@title} % Increase the font size of the title

\vspace{50pt} % Some vertical space between the title and author name

{\large\@author} % Author name
\\\@date % Date

\vspace{40pt} % Some vertical space between the author block and abstract
\end{flushright}
}

%----------------------------------------------------------------------------------------
%	TITLE
%----------------------------------------------------------------------------------------

\title{\textbf{Does reductionism succeed in explaining how we can know things on the basis of testimony?}\\ % Title
} 

\author{\textsc{Stephen Whitenstall} % Author
\\{\textit{The Open University}}} % Institution

\date{April 2015}

%----------------------------------------------------------------------------------------

\begin{document}

\maketitle % Print the title section


%----------------------------------------------------------------------------------------
%	ESSAY BODY
%----------------------------------------------------------------------------------------

\section*{}

In this essay I will address the question of whether reductionism is successful as an explanatory account of how people can know things on the basis of testimony. I will consider versions of anti-reductionism and reductionism from Thomas Reid, David Hume and Elizabeth Fricker, and assess whether Paul Faulkner's and Alan Millar's approaches to testimony are forms of reductionism. Finally I will conclude that different forms of reductionism can only provide partial explanations for different types of testimony.

\vspace{10pt}

The philosophical debate around testimony is broadly concerned with whether testimony can be knowledge in itself or whether it can only be a conveyance of knowledge. A distinction can be drawn in this context between the primacy of knowledge as a true and direct representation of things in themselves and subsidiary belief as being true by virtue of being justified by reasons or having a causal connection to the thing in itself (Goldman, 2005) \cite{Goldman:2005}. The most general definition of reductionism in respect of testimony is the view that testimony cannot be knowledge in itself, but rather must be “defined or explained” by some subsidiary source of information (p.194, Price/Chimmiso, 2014) \cite{Price/Chismisso:2014}
%------------------------------------------------

\vspace{10pt}

\section*{Conclusion}

Fusce in nibh augue. Cum sociis natoque penatibus et magnis dis parturient montes, nascetur ridiculus mus. In dictum accumsan sapien, ut hendrerit nisi. Phasellus ut nulla mauris. Phasellus sagittis nec odio sed posuere. Vestibulum porttitor dolor quis suscipit bibendum. Mauris risus lectus, cursus vitae hendrerit posuere, congue ac est. Suspendisse commodo eu eros non cursus. Mauris ultrices venenatis dolor, sed aliquet odio tempor pellentesque. Duis ultricies, mauris id lobortis vulputate, tellus turpis eleifend elit, in gravida leo tortor ultricies est. Maecenas vitae ipsum at dui sodales condimentum a quis dui. Nam mi sapien, lobortis ac blandit eget, dignissim quis nunc.


%----------------------------------------------------------------------------------------
%	BIBLIOGRAPHY
%----------------------------------------------------------------------------------------

\bibliographystyle{unsrt}
\bibliography{2015-04-Does-reductionism-succeed-in-explaining-how-we-can-know-things-on-the-basis-of-testimony}

%----------------------------------------------------------------------------------------

\end{document}